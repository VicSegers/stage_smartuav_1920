A dynamic environment is an unpredictable environment, the opposite of a static environment. It containts elements that can change over time. Some examples of these elements are walking humans, an automatic door, or light intensity. Therefore, the environment has to be dealt with as little presumptions as possible to act as if its completely unknown. This ensures a more general solution that works in most environments.

This dynamic environment impacts the \acs{uav}'s method of environment interpretation and its navigation in it. Because of the movement of machines, objects and humans, the \acs{uav} must update its path in real time to avoid these. The change in light intensity or colors in the environment state that the \acs{uav} must interpret its surroundings in a general sense. If not, a loop closure (explained in the mapping and localization section) might not occur when in reality needed.
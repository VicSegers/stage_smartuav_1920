The center of expertise Smart \acs{ict} of Hogeschool PXL consists of 21 all\hyp{}round employees and bundles their knowledge of \acs{ict} (software, project management, software architecture, systems) and electronics (focus on hardware and embedded software). The link with education is ensured via the new department PXL\hyp{}Digital, which besides the bachelor programs applied informatics and electronics\hyp{}\acs{ict} also represents graduate programs \acl{iot}, Programming, and Systems \& Networks, with about 1500 students in total.

Smart \acs{ict} is pursuing a double course: on the one hand, efforts are being made in many vertical domains, such as \acs{vr}/\acs{ar}, \ac{iot}, Blockchain, and \acl{ai} \& Robotics; on the other hand, horizontal support is offered to other centers of expertise, through the development of mobile or web\hyp{}based applications. Smart \acs{ict} offers support to partners from various sectors by responding to practical questions about \acs{ict} advice for companies, organizations, and smart cities. The three areas in which Smart \acs{ict} is a priority are \acs{vr} and \acs{ar}, \acl{iot} and \acl{ai} \& Robotics. Finally, Smart \acs{ict} has set itself the goal of evaluating the use of new technologies and transferring these insights to specific target groups, such as the construction sector, education, the retail sector, or the healthcare sector.

I chose Smart \acs{ict} because of my interest in \acl{ai} \& Robotics. It also allowed me to continue working on a project I initiated during my bachelor program. Smart \acs{ict} is an expertise center where research is central, which I consciously chose for transferring to a master's program.
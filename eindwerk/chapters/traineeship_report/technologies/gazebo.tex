% http://www.gazebosim.org/tutorials?cat=guided_b&tut=guided_b1
% https://spectrum.ieee.org/automaton/robotics/robotics-software/latest-version-of-gazebo-simulator
Gazebo is an open-source \acs{3d} dynamic simulator. It can simulate populations of robots in complex environments with high accuracy and efficiency. Gazebo is similar to game engines, but with a much higher degree of fidelity. Sensors simulated in this environment use this fidelity to function almost the same as they would in the real world. \cite{gazebo_beginner_overview} \cite{ackerman2016gazebo}

% http://gazebosim.org/tutorials?tut=ros_overview
% http://gazebosim.org/tutorials?tut=ros_control
Gazebo is able to be connected with \acs{ros} and be used as a replacement of the real world. The connection with \acs{ros} is realized through a set of \acs{ros} packages named \textit{gazebo\_ros\_pkgs}.
This set contains a package that stores all messages and service data structures needed for interacting with Gazebo. Another package provides robot\hyp{}independent Gazebo plugins for sensors, motors, and dynamic components. Also, a package that allows for interfacing Gazebo with \acs{ros}. \cite{gazebo_ros_overview} \cite{gazebo_ros_control}

Gazebo is the standard for simulation when developing \acs{ros} projects because of its great compatibility with \acs{ros}. That's why Gazebo is used in this project. The main reasons for developing in a simulation are safety, consistency, economic, and testing purposes. A drone can be a dangerous machine and expensive machine. If testing and something went wrong the drone could be damaged, or even worse a human could get hurt. That's why Gazebo is a great option for rapid robot prototyping.


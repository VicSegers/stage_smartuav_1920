% http://wiki.ros.org/ROS/Introduction
\ac{ros} is an open\hyp{}source, meta\hyp{}operating system for robots. A meta\hyp{}operating system provides services expected from an operating system, including hardware abstraction, low\hyp{}level device control, implementation of commonly\hyp{}used functionality, message\hyp{}passing between processes, and package management. \acs{ros} also provides tools and libraries for obtaining, building, writing, and running code across multiple computers. \cite{ros_introduction}

% https://books.google.be/books?id=g3JQDwAAQBAJ&pg=PR20&lpg=PR20&dq=ros%27s+primary+goal&source=bl&ots=8_EVzhiHu_&sig=ACfU3U31jyhZZJEwsmE6w5VWPFsMAHL1rA&hl=nl&sa=X&ved=2ahUKEwioitfZwPnoAhWDyKQKHXF-C7EQ6AEwA3oECAgQAQ#v=onepage&q=ros's%20primary%20goal&f=false
% A Systematic Approach to Learning Robot Programming with ROS
The primary goal of \acs{ros} is to support code reuse in robotics research and development. \acs{ros} is a distributed framework of processes - also called nodes - that enables executables to be individually designed and loosely coupled at runtime. By grouping these processes, packages are formed. These packages can be easily shared and distributed. By supporting a federated system of code repositories, \acs{ros} enables the distribution of collaboration. This design enables independent decisions about development and implementation, but can be brought together with the \acs{ros} infrastructure tools. \cite{newman2017systematic}

% http://wiki.ros.org/Messages
The nodes of \acs{ros} communicate with each other by publishing messages to topics. A message is a simple data structure, consisting of typed fields. The standard primitive types (integer, floating\hyp{}point, strings, etc.) are supported in these messages, as are arrays of the primitive types. \acs{ros} allows custom defined messages to be sent over its network. Other nodes can subscribe to topics and receive all messages sent to those topics. When a message is received, a callback is triggered that handles the message or acts on something. The \acs{ros} Master node provides names and registration services for the rest of the nodes in the system. This is how a single node can find another. \cite{ros_messages}

In this project, \acs{ros} is used as a communication medium. It allows the chosen technologies to talk with another in a reliable and standardized manner. \acs{ros} can be considered as the glue that combines the different technologies and creates a whole.
Foremost, I would like to express my sincere gratitude to my promotor Tim Dupont for the continuous support and feedback throughout the internship. Besides my promotor, I would like to thank all employees of Smart ICT for making this internship possible and for their guidance.

I am ever so grateful for this oppertunity given to me by PXL University of Applied Sciences and Arts and Smart ICT. Many of my skills are improved during these twelve weeks. Aswell as I have gotten a taste of a real work environment. During this whole internship, the communication from PXL University and Smart ICT was flawless. Even during the COVID\hyp{}19 period, all guideliness and measurements concerning the virus were clearly and rapidly informed to everyone.

At last, I give thanks to my fellow interns at Smart ICT that assisted me during my internship: Bavo Knaeps and Borcherd van Brakell van Wadenoyen en Doorwerth. If it were not for them, this internship would not have turned out as great as it did.
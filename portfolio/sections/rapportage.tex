\subsection{Wekelijkse rapportage}
  % Deze tabel wordt tijdens de stageperiode voor iedere week ingevuld. Met persoonlijke reflectie bedoelen we: Wat ging goed, wat ging minder goed, wat heb je geleerd, in welke mate heeft de opleiding je kunnen helpen bij het uitvoeren van de opdracht. Welke waren je valkuilen? Wat is de meerwaarde voor het bedrijf? Wat is de meerwaarde voor het team (stageteam en bedrijfsteam)? Welke softskills heb je bijgeleerd? Wat zou je anders doen als je de stage(week) opnieuw zou beginnen? Wat zou beter gekund hebben? Welke lessen heb je geleerd ? Op welke punten ga je je de volgende keer focussen en hoe ga je dit aanpakken?

  \begin{tabularx}{\textwidth}{| l | X |}
    \hline
    Datum: & 24/02/2020 - 28/02/2020\\
    \hline
    Geplande taken: &
    \begin{outline}
      \1 Workstation installeren
      \1 Bestaande code IT-Project omvormen
      \1 SLAM research/begin implementatie (RTAB-Map)
      \1 Gazebo werelden maken
      \1 Modellen iris UAV aanpassen
    \end{outline}\\
    \hline
    Stand van zaken: & De meeste van de geplande taken zijn voltooid. De research naar SLAM nemen we mee naar volgende week. Er zullen ook nog SLAM-algoritmen geïmplementeerd moeten worden, maar dit is wisten we op voorhand. Ook moeten we de iris UAV modellen nog aanpassen, Hiervooor hebben we een issue gepost op het forum van PX4.\\
    \hline
    Problemen en knelpunten: & Iris UAV modellen aanpassen.\\
    \hline
    Oplossingen: & Issue gepost op het forum van PX4.\\
    \hline
    Persoonlijke reflectie: & TODO\\
    \hline
    Planning volgende week: & 
    \begin{outline}
      \1 RTAB-Map verder implementeren
      \1 Onderzoek SLAM-algoritmen
      \1 Implementatie andere algoritmen
    \end{outline}\\
    \hline
  \end{tabularx}

  \begin{tabularx}{\textwidth}{| l | X |}
    \hline
    Datum: & 02/03/2020 - 06/03/2020\\
    \hline
    Geplande taken: &
    \begin{outline}
      \1 TODO
    \end{outline}\\
    \hline
    Stand van zaken: & TODO\\
    \hline
    Problemen en knelpunten: & TODO\\
    \hline
    Oplossingen: & TODO\\
    \hline
    Persoonlijke reflectie: & TODO\\
    \hline
    Planning volgende week: & 
    \begin{outline}
      \1 TODO
    \end{outline}\\
    \hline
  \end{tabularx}

\subsection{Eindrapportage}
  % De eindrapportage wordt aan het einde van de stageperiode ingevuld. Hierin worden de volgende punten aangehaald:
    % • Opgedane ervaring
    % • Verloop van het project
    % • Gesignaleerde problemen
    % • Gekozen oplossing
    % • Persoonlijke reflectie
    % • Eindbesluit (het eindresultaat)
  % Dit is uiteindelijk ook je besluitvorming voor je eindwerk.
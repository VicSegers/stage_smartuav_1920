\begin{tabularx}{\textwidth}{| l | X |}
  \hline
  Datum: & 23/03/2020 - 27/03/2020\\
  \hline
  Geplande taken: &
  \begin{compactitem}
    \item Andere outdoor wereld zoeken
    \item Parameters RTAB-Map tweaken
    \item Onderzoek path finding algoritmen
    \item Onderzoek explore algoritmen
  \end{compactitem}\\
  \hline
  Stand van zaken: & RTAB-Map gebruikt nu alle punten punten om een octomap te maken. Path planning kan nu dus ge\"implementeerd worden. We maken momenteel geen gebruik van een LiDAR, enkel een camera.\\
  \hline
  Problemen en knelpunten: & Het aanpassen van de diepte van RTAB-Map, dus alle punten wat het gebruikt om een map te maken. En de meeste algoritmen over path finding of exploring zijn in 2D.\\
  \hline
  Oplossingen: & Een issue gepost op de rtabmap\_ros repository, geen antwoord op gekregen en uiteindelijk het zelf gevonden. Voorlopig nog geen oplossing gevonden voor de 2D algoritmen over exploring, bij path finding kunnen we waarschijnlijk bestaande omvormen naar 3D.\\
  \hline
  Persoonlijke reflectie: & Omdat we zeer lang hebben gezeten op het tweaken van RTAB-Map lijkt het alsof we niet veel gepresteerd hebben, maar dit was zeker nodig voor het vervolg van het project, dus we konden dit niet omzeilen.\\
  \hline
  Planning volgende week: &
  \begin{compactitem}
    \item Implementatie path finding
    \item Onderzoek explore algoritme
    \item Implementatie zoeken naar QR-code
    \item Implementatie path execution van bekomen pad
  \end{compactitem}\\
  \hline
\end{tabularx}
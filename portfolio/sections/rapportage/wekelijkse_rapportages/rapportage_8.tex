\begin{tabularx}{\textwidth}{| l | X |}
  \hline
  Datum: & 27/04/2020 - 01/05/2020\\
  \hline
  Geplande taken: &
  \begin{compactitem}
    \item Afwerking portfolio en taalversie eindwerk
    \item Andere aansturing van UAV onderzoeken
    \item Verkeerde loopclosures van RTAB-Map fixen
  \end{compactitem}\\
  \hline
  Stand van zaken: & De UAV kan een vast pad vliegen en ondertussen SLAM'en. Tijdens het onderzoek naar path planning wat de volgende stap is, moeten er nog een paar bugs gefixt worden die een direct probleem vormen.\\
  \hline
  Problemen en knelpunten: & De UAV vliegt niet in een rechte lijn van punt naar punt, maar met een horizontale afwijking. RTAB-Map geeft verkeerde loopclosures door een gebrek aan textures.\\
  \hline
  Oplossingen: & De UAV aansturen aan de hand van velocity in plaats van position. Toevoeging objecten met verschillende textures.\\
  \hline
  Persoonlijke reflectie: & De aansturing op basis van velocity geeft dezelfde output als het vliegen op punten. De toevoeging van de objecten maakt de loopclosures minder frequent, maar ze zijn er nog.\\
  \hline
  Planning volgende week: &
  \begin{compactitem}
    \item Optimalisaties RTAB-Map
    \item Onderzoek missions
  \end{compactitem}\\
  \hline
\end{tabularx}
\begin{tabularx}{\textwidth}{| l | X |}
  \hline
  Datum: & 04/05/2020 - 08/05/2020\\
  \hline
  Geplande taken: &
  \begin{compactitem}
    \item Optimaliseren RTAB-Map
    \item Onderzoek missions
    \item UAV 'smoother' laten vliegen
    \item Implementatie RTAB-Map optimalisatie GTSLAM
  \end{compactitem}\\
  \hline
  Stand van zaken: & De UAV vertrekt en stopt te abrupt waardoor RTAB-Map soms niet kan volgen. RTAB-Map optimaliseren voor een hogere performance.\\
  \hline
  Problemen en knelpunten: & Vertrekken en draaien van UAV moet vlotter.\\
  \hline
  Oplossingen: & De UAV laten vliegen op missions is niet mogelijk in dit project, het vliegen op velocity is geen optie dus de snelheid en yaw rate lager zetten. De UAV stil laten hangen tijden het aanpassen van de yaw geeft een rustigere vlucht voor RTAB-Map.\\
  \hline
  Persoonlijke reflectie: & Doordat de UAV moet stilhangen om zijn yaw aan te passen is het niet super realistisch. Maar door restricties van het project en het moeten werken in een simulatie moeten deze implementaties gedaan worden. Het implementeren van path planning is nu noodzakelijk.\\
  \hline
  Planning volgende week: &
  \begin{compactitem}
    \item Implementeren path planning
  \end{compactitem}\\
  \hline
\end{tabularx}
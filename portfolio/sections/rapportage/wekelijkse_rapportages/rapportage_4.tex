\begin{tabularx}{\textwidth}{| l | X |}
  \hline
  Datum: & 16/03/2020 - 20/03/2020\\
  \hline
  Geplande taken: &
  \begin{compactitem}
    \item Research, uittesten en implementatie RTAB-Map
    \item AIHub met textures maken
    \item Outdoor wereld SLAM'en
    \item Onderzoek path finding
  \end{compactitem}\\
  \hline
  Stand van zaken: & De implementatie van RTAB-Map is gelukt, er worden enkel te weinig referentiepunten gevonden in onze huidige wereld van de AIHub. De outdoor wereld heeft problemen met loopclosures en verliest af en toe ook zijn punten.\\
  \hline
  Problemen en knelpunten: & Te weinig textures in AIHub (te weinig referentie punten voor SLAM), TF-links waren verkeerd maar geen regels of conveties gevonden, te veel loopclosures zonder gps en te weinig met.\\
  \hline
  Oplossingen: & Exporteren AIHub met textures van blender, TF-links met trail-and-error opgelost en waarschijnlijk andere outdoor wereld zoeken waar er meer unieke omgevingen zijn.\\
  \hline
  Persoonlijke reflectie: & Door het twee keer moeten verhuizen wegens COVID-19, is er deze week wat minder werk geleverd als ik gehoopt had. We hebben ook te lang vast gezeten met de textures van de AIHub en lang op een medestudent moeten wachten die het uiteindelijk gemaakt heeft.\\
  \hline
  Planning volgende week: &
  \begin{compactitem}
    \item Andere outdoor wereld zoeken
    \item Parameters RTAB-Map finetunen
    \item Onderzoek naar path finding
  \end{compactitem}\\
  \hline
\end{tabularx}
% In de situatieschets geef je aan in welke omgeving het project zich afspeelt. Omschrijf de aard van het stagebedrijf en de activiteiten van het stagebedrijf. Wat is het doel van de organisatie? Welke producten of diensten worden er geleverd? Welke zijn de Unique Selling Points van het bedrijf op vlak van IT (in welk IT domein zijn ze gespecialiseerd, op welk vlak van de IT business willen ze zich profileren, ...)? Wat is jouw motivering van het gekozen domein, het gekozen project en het gekozen stagebedrijf?

Het expertisecentrum Smart ICT van Hogeschool PXL bestaat uit een team van 21 allround medewerkers en bundelt de kennis van ICT (software, project management, software architectuur, systemen) en elektronica (focus op hardware en embedded software). De link met het onderwijs is verzekerd via het nieuwe departement PXL-Digital, waarin naast de bacheloropleidingen toegepaste informatica en elektronica-ICT ook de graduaatsopleidingen Internet of Things, Programmeren en Systemen \& Netwerken vertegenwoordigd zijn, in totaal een 1500-tal studenten.

Smart ICT vaart een dubbele koers: enerzijds wordt er ingezet op een aantal verticale domeinen, zoals VR/AR, Internet of Things (IoT), Blockchain en Artifici\"ele Intelligentie \& Robotica; anderzijds wordt er horizontaal ondersteuning geboden aan andere expertisecentra, door de ontwikkeling van mobiele applicaties of web-gebaseerde toepassingen. Smart ICT biedt partners uit diverse sectoren ondersteuning door in te spelen op praktische vragen rond ICT-advies voor bedrijven, organisaties en smart cities. De drie domeinen waar Smart ICT prioritair op inzet zijn VR en AR, Internet of Things en Artifici\"ele Intelligentie \& Robotica. Smart ICT heeft zich tot slot als doel gesteld om de inzet van nieuwe technologie\"en te evalueren en deze inzichten over te dragen naar specifieke doelgroepen, zoals de bouwsector, het onderwijs, de retailsector of de zorgsector.

Zelf heb ik voor Smart ICT gekozen wegens mijn interesse in Artifici\"ele Intelligentie \& Robotica. Het gaf mij ook de mogelijkheid om aan een project verder te werken waar ik al een aanzet aan heb gegeven tijdens het IT-project. Smart ICT is een expertisecentrum waar onderzoek centraal ligt, hier heb ik bewust voor gekozen. Deze keuze kwam voort omdat ik volgend jaar door wil schakelen naar een masteropleiding.
% Voeg hier je planning toe voor de 12 weken van je stageperiode. Gebruik hiervoor de technieken die je hebt aangeleerd in het opleidingsonderdeel Business Flow Advanced 1. Indien je stagebedrijf een andere methode hanteert voor het opstellen van een planning, dien je deze methode over te nemen.

% Effectief verloop van het project

% Bijkomend moet op dezelfde wijze het effectieve verloop van het project weergegeven worden.
% Zo zal je waarschijnlijk ook merken dat de planning, door omstandigheden, niet voor honderd procent gevolgd kan worden. Maar een planning geeft jezelf en je promotoren wel een goed beeld van nog moet gebeuren: in welke fase je op een bepaald moment zit of zou moeten zitten. Dit kan heel eenvoudig door een bijkomende balk te gebruiken in een andere kleur.

% Wanneer jullie met meerderen aan een project werken, moet uit de planning duidelijk afgeleid kunnen worden welke taken voor welke student gelden.

Zie bijhorend bestand: \textit{Vic\_Segers\_tijdsplanning.xlsx}
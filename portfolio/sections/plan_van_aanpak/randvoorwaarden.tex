% Welke afspraken heb je gemaakt met:
  % • het stagebedrijf, de bedrijfspromotor (werkregime, gedrag, …);
  % • de school, de hogeschoolpromotor (overlegfrequentie, manier van rapporteren, …);
  % • jezelf (vooropgestelde actiepunten, inzet, doel, …);
  % • de medestudenten indien je met meerdere studenten aan een project werkt.

  \begin{outline}
  % TODO: aanvullen!
  \1 \textbf{Beslissingen:}
    \2[] De wijze waarop besluiten tot stand komen, bijvoorbeeld: je bedenkt een aantal mogelijke oplossingen met zijn voor- en nadelen en de bedrijfspromotor maakt hier een keuze uit.
  \1 \textbf{Beperkingen:}
    \2[] Omschrijf eventuele beperkingen waarmee je rekening moet houden. Dit kunnen beperkingen op materiaal, tijd, personeel, ruimte, kennis, e.d. zijn.
  \1 \textbf{Kritische succesfactoren:}
    \2 \hl{Workstation (incl. scherm, muis en toetsenbord)}
    \2[] Wat heb je zeker nodig om het project te laten slagen (bijvoorbeeld werkruimte, apparatuur, ondersteuning...). Dit kunnen eventueel ook middelen of werkruimten buiten het bedrijf zijn.
  \1 \textbf{Onzekerheden:}
    \2[] Het bereiken van de doelstellingen kan voor sommige projecten, afhankelijk van de beperkingen en de kritische succesfactoren, nog een vraagteken zijn. Vermeld hier deze onzekerheden.
  \1 \textbf{Afspraken:}
    \2 Studenten houden zich aan de glijdende werkuren van \stagebedrijf{} (I.e. 38u per week, starten $\leq$ 9 am, minimaal 30 min middagpauze)
    \2 Alles wordt ontwikkeld op eigen laptop of op de voorziene workstations, tijdens de werkuren pushen naar een met de begeleider afgesproken repository
      \3 Version Control verplicht te gebruiken voor Smart-ICT
      \3 Github repo te voorzien door begeleider
    \2 Voortgang bewaken door frequent stand-up meeting te doen
      \3 Dagelijks invullen in $\#$standup kanaal op Slack workspace vóór 09u15
      \3 Fysieke stand-up met alle stagairs $\rightarrow$ 1x per week op vrijdag
    \2 1x per sprint presentatie van resultaten naar de begeleider toe (sprint planning)
    \2 Insturingen
      \3 Portfolio’s en andere communicatie steeds via EPOS naar begeleider PXL 
      \3 Deadlines voor teksten op EPOS respecteren! 
      \3 \colorbox{yellow}{Geel markeren wat nieuw of aangepast is}
    \2 Solliciteren tijdens stageperiode
      \3 Flexibel
      \3 Afstemmen met begeleiders en team
      \3 Tijd inhalen
      \3 Andere afspraken uit PPT stagebegeleiding
    \2 Geen shenanigans
      \3 Geen speeltuin
      \3 Geen boksring
      \3 etc.
\end{outline}
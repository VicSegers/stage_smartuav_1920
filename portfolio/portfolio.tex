\def\student{Vic Segers}
\def\bedrijfspromotor{Tim Dupont}
\def\hogeschoolpromotor{Tim Dupont}
\def\stagebedrijf{PXL Expertisecentrum Smart ICT}

\documentclass[a4paper]{article}
\usepackage{./PXLStageportfolio}

\begin{document}
  \pagenumbering{gobble}  % No page numbering
  \titlepage{\stagebedrijf}{\student}{\bedrijfspromotor}{\hogeschoolpromotor}

  \begingroup
    \hypersetup{linkcolor=black}  % Black links in table of contents
    \tableofcontents
  \endgroup

  \pagenumbering{arabic}  % Start page numbering

  \section{Informatiefiche}
    \begin{tabbing}
      Student: \textbf{\student}\\
      \\
      ~~~~~~ \= vic.segers10@gmail.com\\
      \> +32 476 62 57 66\\
      \> AON - AI \& Robotics
    \end{tabbing}
    \begin{tabbing}
      Bedrijf stageplek: \textbf{\stagebedrijf}\\
      \\
      ~~~~~~ \= Elfde-liniestraat 24\\
      \> B-3500 Hasselt
    \end{tabbing}
    \begin{tabbing}
      Bedrijfspromotor: \textbf{Dhr. \bedrijfspromotor}\\
      \\
      ~~~~~~ \= +32 495 46 55 25\\
      \> tim.dupont@pxl.be
    \end{tabbing}
    \begin{tabbing}
      Hogeschoolpromotor: \textbf{Dhr. \hogeschoolpromotor}
    \end{tabbing}
    Stageproject: \textbf{Navigating Smart UAV systems in dynamic environments}
    \begin{addmargin}[5ex]{0pt}
      Het doel van de stage is om een generieke architectuur te ontwikkelen voor verschillende UAV's.
      Er zulen ook enkele demonstraties uitgewerkt worden op dit systeem met toepassingen van SLAM-algoritmen.
    \end{addmargin}

  \section{Plan van aanpak}
    \subsection{Situatieschets stagebedrijf + motivatie}
      % In de situatieschets geef je aan in welke omgeving het project zich afspeelt.
      % Omschrijf de aard van het stagebedrijf en de activiteiten van het stagebedrijf.
      % Wat is het doel van de organisatie? Welke producten of diensten worden er
      % geleverd? Welke zijn de Unique Selling Points van het bedrijf op vlak van IT
      % (in welk IT domein zijn ze gespecialiseerd, op welk vlak van de IT business
      % willen ze zich profileren, ...)? Wat is jouw motivering van het gekozen domein,
      % het gekozen project en het gekozen stagebedrijf?
      Het project zal door 2 stagestudenten ontwikkeld worden. Iedere student krijgt
      een specifieke taak, maar er gaat veel worden samen gewerkt. De stage
      gaat door in de AI Hub op de Corda Campus.\par
      \stagebedrijf{} ontwikkelt en onderzoekt allerlei slimme toepassingen van de
      emerging techonologies en bundelt praktijkgericht onderzoek en dienstverlening
      op het snijvlak van IT en elektronica. Het spitst zich toe op mobiele technologie
      in de breedste zin van het woord en zoekt naar slimme oplossingen voor
      objecten en gemeenschappen: van slimme sensoren, uurwerken, telefoons en tablets,
      tot slimme gebouwen en smart cities. Daarbij staat de toepasbaarheid en
      inzetbaarheid van ontwikkelende toepassingen in de praktijk steeds voorop.\par
      Ik heb voor \stagebedrijf{} gekozen omdat ik er de kans kreeg om verder onderzoek
      te doen waaraan ik tijdens mijn IT-Project ook al aan heb gewerkt. Ook het kiezen
      van een meer onderzoekende stage leek mij niet slecht om na mijn bachelor door te
      schakelen naar een master.

      \newpage

    \subsection{Probleemstelling(en)}
      % Waarom er iets moet worden geproduceerd, blijkt uit de probleemstelling. Wat
      % is het probleem dat je moet gaan oplossen? Waarom werd het project opgestart,
      % in welke fase bevindt het project zich op het moment van de stage,...? Wat er
      % geproduceerd gaat worden, wordt in de doelstelling uiteengezet.
      Het project is opgestart om een basis architectuur te hebben zodat er makkelijk en
      snel een ontwikkelingsomgeving ter beschikking is. Verder wordt er ook onderzoek gedaan
      naar verschillende SLAM algoritmes en deze implementeren. Het gaat over indoor en outdoor
      implementaties, ik ga mij vooral focussen op de indoor algoritmes.\par
      Het project heeft al meerdere voorafgaande fasen gehad. Zo heeft een IT-project
      de architectuur al geïmplementeerd in een specifieke toepassen. Er gaat dus nu een meer generieke
      structuur moeten geschreven worden. Er is ook al een stage rond het onderzoek van SLAM-algoritmen
      gegaan, dit moet verder uitgebreidt worden en specifieke toepassingen ervoor ontwikkelen.

    \newpage

    \subsection{Doelstelling(en)}
      % Door middel van een doelstelling wordt het gewenste eindresultaat van het project
      % omschreven. Onder omschrijving verstaan we een bondige en relevante
      % beschrijving van de doelstellingen die je wil verwezenlijken, de methodiek die je
      % kan/gaat aanwenden, ...
      Er moet een architectuur gemaakt worden voor multi-agentsystemen. Inclusief documentatie
      hoe deze gebruikt kan worden. Het gaat ook gebruikt worden om verschillende toepassingen
      uit te werken. Deze toepassingen implementeren verschillende SLAM-algoritmen waar tijdens
      de stage onderzoek naar gedaan wordt. Er wordt een onderscheid gemaakt tussen indoor en outdoor
      toepassingen, dus er zijn minimaal 2 uitwerkingen die opgeleverd moeten worden.\par
      Het onderzoek en de wijze waarop wordt uitgeschreven in een thesis.

    \newpage

    \subsection{Randvoorwaarden}
      % Welke afspraken heb je gemaakt met:
      % • het stagebedrijf, de bedrijfspromotor (werkregime, gedrag, …);
      % • de school, de hogeschoolpromotor (overlegfrequentie, manier van rapporteren, …);
      % • jezelf (vooropgestelde actiepunten, inzet, doel, …);
      % • de medestudenten indien je met meerdere studenten aan een project werkt.
      \begin{outline}
        \1 \textbf{Beslissingen:}
          \2[] De wijze waarop besluiten tot stand komen, bijvoorbeeld: je bedenkt een aantal
          mogelijke oplossingen met zijn voor- en nadelen en de bedrijfspromotor maakt hier een keuze uit.
        \1 \textbf{Beperkingen:}
          \2[] Omschrijf eventuele beperkingen waarmee je rekening moet houden. Dit kunnen
          beperkingen op materiaal, tijd, personeel, ruimte, kennis, e.d. zijn.
        \1 \textbf{Kritische succesfactoren:}
          \2[] Wat heb je zeker nodig om het project te laten slagen (bijvoorbeeld werkruimte, apparatuur,
          ondersteuning...). Dit kunnen eventueel ook middelen of werkruimten buiten het bedrijf zijn.
        \1 \textbf{Onzekerheden:}
          \2[] Het bereiken van de doelstellingen kan voor sommige projecten, afhankelijk van de beperkingen
          en de kritische succesfactoren, nog een vraagteken zijn. Vermeld hier deze onzekerheden.
        \1 \textbf{Afspraken:}
          \2 Studenten houden zich aan de glijdende werkuren van \stagebedrijf{} (I.e. 38u per week, starten $\leq$
          9 am, minimaal 30 min middagpauze)
          \2 Alles wordt ontwikkeld op eigen laptop of op de voorziene workstations, tijdens de werkuren pushen
          naar een met de begeleider afgesproken repository
            \3 Version Control verplicht te gebruiken voor Smart-ICT
            \3 Github repo te voorzien door begeleider
          \2 Voortgang bewaken door frequent stand-up meeting te doen
            \3 Dagelijks invullen in $\#$standup kanaal op Slack workspace vóór 09u15
            \3 Fysieke stand-up met alle stagairs $\rightarrow$ 1x per week op vrijdag
          \2 1x per sprint presentatie van resultaten naar de begeleider toe (sprint planning)
          \2 Insturingen
            \3 Portfolio’s en andere communicatie steeds via EPOS naar begeleider PXL 
            \3 Deadlines voor teksten op EPOS respecteren! 
            \3 \colorbox{yellow}{Geel markeren wat nieuw of aangepast is}
          \2 Solliciteren tijdens stageperiode
            \3 Flexibel
            \3 Afstemmen met begeleiders en team
            \3 Tijd inhalen
            \3 Andere afspraken uit PPT stagebegeleiding
          \2 Geen shenanigans
            \3 Geen speeltuin
            \3 Geen boksring
            \3 etc.
      \end{outline}

    \newpage

  \subsection{Tijdsplanning}
    % Voeg hier je planning toe voor de 12 weken van je stageperiode.\\
    % Gebruik hiervoor de technieken die je hebt aangeleerd in het opleidingsonderdeel
    % Business Flow Advanced 1.\\
    % Indien je stagebedrijf een andere methode hanteert voor het opstellen van een
    % planning, dien je deze methode over te nemen.\par
    % \underline{Effectief verloop van het project}\par
    % Bijkomend moet op dezelfde wijze het effectieve verloop van het project
    % weergegeven worden.\\
    % Zo zal je waarschijnlijk ook merken dat de planning, door omstandigheden, niet voor
    % honderd procent gevolgd kan worden. Maar een planning geeft jezelf en je promotoren
    % wel een goed beeld van nog moet gebeuren: in welke fase je op een bepaald moment zit
    % of zou moeten zitten. Dit kan heel eenvoudig door een bijkomende balk te gebruiken in
    % een andere kleur.\par
    % Wanneer jullie met meerderen aan een project werken, moet uit de planning
    % duidelijk afgeleid kunnen worden welke taken voor welke student gelden.

  \subsection{Bronnen}
    % Er dient altijd expliciet te worden aangegeven op welke bronnen en naslagwerken
    % het eindwerk berust. Door systematisch hierin de bronnen te vermelden verliest
    % men achteraf niet te veel tijd om de bronnen terug te gaan opzoeken. Het kan
    % om schriftelijke, mondelinge een elektronische bronnen gaan.\par
    % Voor de omschrijving gelden enkele vaste regels die vermeld staan in de
    % cursus op BlackBoard.

  \section{Rapportage}
    \subsection{Wekelijkse rapportage}
      % Deze tabel wordt tijdens de stageperiode voor iedere week ingevuld.
      % Met persoonlijke reflectie bedoelen we:
      % Wat ging goed, wat ging minder goed, wat heb je geleerd, in welke mate heeft de
      % opleiding je kunnen helpen bij het uitvoeren van de opdracht. Welke waren je
      % valkuilen? Wat is de meerwaarde voor het bedrijf? Wat is de meerwaarde voor het
      % team (stageteam en bedrijfsteam)? Welke softskills heb je bijgeleerd? Wat zou je
      % anders doen als je de stage(week) opnieuw zou beginnen? Wat zou beter gekund hebben?
      % Welke lessen heb je geleerd ? Op welke punten ga je je de volgende keer focussen
      % en hoe ga je dit aanpakken?

      \begin{tabularx}{\textwidth}{| l | X |}
        \hline
        Datum: & 24/02/2020 - 28/02/2020\\
        \hline
        Geplande taken: & 
        \begin{outline}
          \1 Workstation installeren
          \1 Bestaande code IT-Project omvormen
          \1 SLAM research/begin implementatie (RTAB-Map)
          \1 Gazebo werelden maken
          \1 Modellen iris UAV aanpassen
        \end{outline}\\
        \hline
        Stand van zaken: & 
        De meeste van de geplande taken zijn voltooid. De research naar SLAM nemen we mee
        naar volgende week. Er zullen ook nog SLAM-algoritmen geïmplementeerd moeten worden,
        maar dit is wisten we op voorhand. Ook moeten we de iris UAV modellen nog aanpassen,
        Hiervooor hebben we een issue gepost op het forum van PX4.\\
        \hline
        Problemen en knelpunten: & 
        Iris UAV modellen aanpassen.\\
        \hline
        Oplossingen: & 
        Issue gepost op het forum van PX4.\\
        \hline
        Persoonlijke reflectie: & item 62\\
        \hline
        Planning volgende week: & 
        \begin{outline}
          \1 RTAB-Map verder implementeren
          \1 Onderzoek SLAM-algoritmen
          \1 Implementatie andere algoritmen
        \end{outline}\\
        \hline
      \end{tabularx}

    \subsection{Eindrapportage}
      % De eindrapportage wordt aan het einde van de stageperiode ingevuld. Hierin worden de
      % volgende punten aangehaald:\\
      %   • Opgedane ervaring\\
      %   • Verloop van het project\\
      %   • Gesignaleerde problemen\\
      %   • Gekozen oplossing\\
      %   • Persoonlijke reflectie\\
      %   • Eindbesluit (het eindresultaat)\\
      % Dit is uiteindelijk ook je besluitvorming voor je eindwerk.

  \section{Terugkoppelingsformulieren}
    \subsection{Stagebespreking}
      % Telkens wanneer je een bespreing hebt, zowel met de hogeschoolpromotor als
      % de bedrijfspromotor, vul je onderstaand formulier in:\par
      % \begin{tabularx}{\textwidth}{| l | X |}
      %   \hline
      %   Volgnummer: & Bv. Stage2019-01\\
      %   \hline
      %   Datum: & \\
      %   \hline
      %   Besproken punten: & \\
      %   \hline
      %   Besluiten: & \\
      %   Acties: & \\
      %   Afspraken: & \\
      %   \hline
      %   Goedgekeuring & \\
      %   \hline
      % \end{tabularx}
\end{document}
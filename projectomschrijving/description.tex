The center of expertise PXL Smart ICT has an ongoing project to map an unknown environment, in- and outdoors. The mapping is being done by a UAV, an unmanned aerial vehicle, which flies autonomously in its environment. The UAV uses an algorithm called SLAM, simultaneous localization and mapping, that creates a map and localizes itself in this map.

The algorithm behaves the same way a human being would when dropped in an unknown environment. A human being looks around and searches for reference points in that environment as landmarks for its own localization. However UAV uses sensors, instead of senses, to get information of its surroundings and uses this to search for reference points. When flying, it can estimate its movement based on the movement of these reference points.

The technologies used in this project are ROS (Robotic Operating System) in combination with MAVROS (implementation of MAVLink in ROS) to send messages to a PX4, the autopilot. For safety and testing purposes, the entire project is developed in a virtual environment named Gazebo. To make the project scalable, modular and consistent a Multi-Docker System is used.

The project is split into two parts: an architecture and a showcase. The architecture, which can handle multiple UAVs, contains a lightweight system for developers to start working with MAVROS and PX4 in a clean environment simulated in Gazebo. The showcase uses the architecture as a base and adds implementations of some SLAM algorithms. In the showcase the UAV is able to fly autonomously in an unknown environment.

TODO: SLOT
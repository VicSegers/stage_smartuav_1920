The center of expertise PXL Smart ICT has an ongoing project to map an unknown environment, in- and outdoor. The mapping is being done by a UAV, an unmanned aerial vehicle, which flies autonomously in an unknown environment. You can look at it like it’s a person who is dropped in a room they have never been before. The first thing that they do is look around, look at reference points where they can locate themselves from.The UAV will do the same thing. It will use a LiDAR and a depth camera to locate itself in the environment it hasn’t seen before.

The technologies that are used in this project are ROS (Robotic Operating System) in combination with MAVROS (implementation of MAVLink in ROS) to control the UAV, which will be a PX4 autopilot. For safety and testing reasons, the entire project will be done in a virtual environment, which we will use Gazebo and Rviz for. To make it scalable, modular and consistent we will use a Multi-Docker System.

The project will be split up into two parts, an architecture and a showcase. The architecture will contain a lightweight system for developers to start working with MAVROS and PX4 in a clean environment. The showcase will use the architecture, combined with the SLAM algorithms. With these algorithms the UAV is capable of flying in an unknown environment autonomously.
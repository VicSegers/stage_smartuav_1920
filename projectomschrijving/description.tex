The center of expertise PXL Smart ICT, part of the PXL University of Applied Sciences and Arts research department, has an ongoing for enabling IT companies to implement UAV, unmanned aerial vehicle, projects via rapid robot prototyping.

The goals of our project is to update and fine-tune an existing architecture, do research into the realm of simultaneous localization and mapping (SLAM) algorithms and showcases that use the architecture and implement these SLAM algorithms for flying autonomously in an unknown environment.

A SLAM algorithm behaves the same way a human being would when dropped in an unknown environment. A human being opens her or his eyes and looks around in search of reference points in their environment. These reference points are used as landmakrs for their own localization. However, a UAV uses sensors, instead of senses, to get information of its surroundings and uses this to search for reference points. When flying, it can estimate its movement based on the movement of these landmarks.

The technologies used are ROS, MAVROS, PX4, Gazebo and Docker. ROS, Robotic Operating System, is an open source set of software libraries and tools, currently viewed as the standard for developing robotics projects. MAVROS is a ROS implementation of MAVLink, Micro Air Vehicle Link, a lightweight messaging protocol for communicating with drones. The open source autopilot used is a PX4, it translates messages from MAVROS to actual commands for the UAV. Gazebo is a tool to simulate a virtual environment. To make system flexible, modular and consistent a multi-container Docker environment is used.

In the future, the ROS version may be upgraded to the one that will be released at the end of May. Because of the structure of ROS2, upgrading to this would require another architecture.

% bronnen
% https://www.ros.org/
% https://mavlink.io/en/
% https://px4.io/
% https://opensource.com/resources/what-docker
% http://gazebosim.org/
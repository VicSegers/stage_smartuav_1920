The center of expertise PXL Smart ICT, part of the PXL University of Applied Sciences and Arts research department, has an ongoing \hl{project} for enabling IT companies to implement \hl{unmanned aerial vehicle (UAV)} projects via rapid robot prototyping.

\hl{One of the goals of this project is updating and fine-tuning an existing Smart UAV software architecture. Another objective is conducting research into the realm of simultaneous localization and mapping (SLAM) algorithms. Combining previous goals into a showcase where a UAV uses a SLAM algorithm to flying autonomously in an unknown environment.}

A SLAM algorithm behaves the same way a human being would when dropped in an unknown environment. A human being opens her or his eyes and looks around in search of reference points in their environment. These reference points are used as \hl{landmarks} for their own localization. However, a UAV uses sensors, instead of senses, to get information \hl{about} its surroundings and uses this to search for reference points. When flying, it can estimate its movement based on the movement of these landmarks.

\hl{The project uses a couple of technologies. Robotic Operating System (ROS) is an open source set of software libraries and tools, it is used as an abstraction layer between the code and the robots hardware. MAVROS is a ROS implementation of Micro Air Vehicle Link (MAVLink), a lightweight messaging protocol for communicating with drones. MAVROS is used to send commands to the UAV. A PX4 autopilot receives these commands and translates them to actual actions the UAV has to execute. For safety and testing purposes, the entire project is developed in an open source 3D robotics simulator, Gazebo. To make the system flexible, modular and consistent a multi-container docker environment is used.}

% bronnen
% https://www.ros.org/
% https://mavlink.io/en/
% https://px4.io/
% https://opensource.com/resources/what-docker
% http://gazebosim.org/
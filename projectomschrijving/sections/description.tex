The center of expertise PXL Smart ICT, part of the PXL University of Applied Sciences and Arts research department, has an ongoing project for enabling IT companies to implement unmanned aerial vehicle (UAV) projects via rapid robot prototyping. This internship is an integral part of that research project.

One of the goals of this internship project is updating and fine\hyp{}tuning an existing Smart UAV software architecture. Another objective is researching the realm of simultaneous localization and mapping (SLAM) algorithms. Robots utilize these algorithms to create a map using their sensors and at the same time locate themselves within this map. Once a map of a certain area exists, a path planning method can be executed in order to navigate between points. Previous goals and objectives will be combined into a showcase where a UAV makes use of a SLAM method to fly and navigate autonomously in a previously unknown dynamic environment.

A SLAM algorithm behaves the same way a human being would when dropped in an unknown environment. A human being opens their eyes and looks around in search of reference points in their environment. These reference points are used as landmarks for their localization. However, unlike humans who use their senses, a UAV uses sensors to get information about its surroundings and uses this to search for reference points. While flying, it can estimate its position based on the movement of these landmarks.

The internship project uses a combination of diverse technologies. Python is predominately used as the programming language. Between the different hardware components and the controlling software, Robotic Operating System (ROS) is being used. ROS is an open\hyp{}source robotics middleware. Remote controlling the UAV is done by using MAVROS. The controlling software implemented with Python and ROS uses MAVROS to communicate via the MAVLink protocol. MAVLink is a lightweight messaging protocol for communicating with drones and between onboard drone components. An autopilot receives these commands and translates them to actual actions that the UAV has to execute. All autopilots used during the internship are based on the open\hyp{}source PX4 flight control software for drones and other unmanned vehicles. For safety and testing purposes, the entire internship project is developed in an open\hyp{}source 3D robotics simulator, Gazebo. To make the system flexible, modular, and consistent a multi\hyp{}container Docker environment is used.

% Sources
% https://www.ros.org/
% https://mavlink.io/en/
% https://px4.io/
% https://opensource.com/resources/what-docker
% http://gazebosim.org/
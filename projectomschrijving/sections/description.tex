The center of expertise PXL Smart ICT, part of the PXL University of Applied Sciences and Arts research department, has an ongoing project for enabling IT companies to implement unmanned aerial vehicle (UAV) projects via rapid robot prototyping. This internship is an integral part of this research project.

One of the goals of this internship project is updating and fine-tuning an existing Smart UAV software architecture. Another objective is researching the realm of simultaneous localization and mapping (SLAM) algorithms. Robots use these algorithms to create a map using their sensors and \hl{at the same time locate themselves} within this map. Combining previous goals into a showcase where a UAV uses a SLAM algorithm to flying autonomously in an unknown environment.

A SLAM algorithm behaves the same way a human being would when dropped in an unknown environment. A human being opens her or his eyes and looks around in search of reference points in their environment. These reference points are used as landmarks for their localization. However, a UAV uses sensors, instead of senses, to get information about its surroundings and uses this to search for reference points. When flying, it can estimate its movement based on the movement of these landmarks.

The internship project uses a combination of diverse technologies. Python is predominately used as the programming language. Between the different hardware components and the controlling software, Robotic Operating System (ROS) is being used. ROS is an open-source robotics middleware. Remote controlling the UAV is done by MAVROS. The controlling software implemented with Python and ROS uses MAVROS to communicate via the MAVLink protocol. \hl{MAVLink} is a lightweight messaging protocol for communicating with drones and between onboard drone components. \hl{An} autopilot receives these commands and translates them to actual actions the UAV has to execute. \hl{All autopilots used during the internship are based on the open-source PX4 flight control software for drones and other unmanned vehicles.} For safety and testing purposes, the entire internship project is developed in an open-source 3D robotics simulator, Gazebo. To make the system flexible, modular and consistent a multi-container \hl{Docker} environment is used.

% Sources
% https://www.ros.org/
% https://mavlink.io/en/
% https://px4.io/
% https://opensource.com/resources/what-docker
% http://gazebosim.org/